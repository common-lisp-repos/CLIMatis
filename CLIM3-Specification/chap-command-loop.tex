\chapter{Commands and the command loop}

A \emph{command} is a \cl{} function that has a \emph{presentation
  type} associated with each required parameter.  

\Defmacro {define-command} {name lambda-list \body body}

This macro is like the \cl{} macro \code{defun}, except that a
required parameter can be either a symbol naming the parameter or a
list whose first element is the name of the parameter and whose second
element is a \emph{presentation type}.  

The macro \code{define-command} does define an ordinary \cl{} function
with the same name as the command, and that function accepts arguments
in conformance with the given lambda list.  However, the function may
have been altered in some way to allow \clim{} to determine the
presentation types of the arguments. 

To invoke a command, \clim{} first acquires an \emph{action}.  An
action is either simply the \emph{name} of a command, or a list where
the first element is the name of the command and each remaining
element corresponds to an argument of the command.  An element of an
action can be an object that should be supplied to the command
invocation in the corresponding argument position, or in can be the
symbol \code{?} (in the \code{clim3} package) which indicates that the
corresponding argument was not supplied as part of the action. 

To invoke the command, \clim{} must first request values for the
unsupplied arguments in the action.  This is done by calling the
generic function \code{accept} with the presentation type of the
corresponding parameter of the command.  When all unsupplied arguments
have been acquired this way, the \cl{} function corresponding to the
command is \emph{called} by using \code{apply} passing the command
name and the list of arguments.

An action can be supplied in different ways.  If the application has a
\emph{command processor} associated with it, then an action can be
supplied by a sequence of keystrokes, either in the form of a
\emph{command line interface} or in the form of \emph{Emacs-style
  keystrokes}, or a combination of the two.  An action can also be
supplied by a \emph{gadget} that has been created for that purpose.
The gadget can be an area to click on, or a menu item.   When the
application does not have a command processor, then the only way of
supplying actions is by this last method. 

Executing a \clim{} application involves calling the \clim{}
\emph{command loop}.  The command loop repeatedly acquires an action,
acquires the unsupplied arguments, and invokes the command.  

During the execution of the command loop, event processing continues
as usual.  It is therefore possible to resize windows, use scroll bars
etc.  If, as a result of executing a command, the hierarchy of zones
is modified in some way, the visible part of this hierarchy is updated
automatically.

%%  LocalWords:  unsupplied resize
