\chapter{Presentations and presentation types}
\label{chap-presentations}

One of the unique features of \clim{}, and one that was inherited from
its predecessor \climtwo{} is the concept of a \emph{presentation
  type}.  Using presentation types allows for a more modular and
flexible user interface.  In this chapter, we use several examples of
typical interaction scenarios based on presentations and presentation
types. 

When an application displays objects of the application model as
potentially clickable figures or images, it does so using a
\emph{presentation}.  A presentation is a zone subclass that contains
the application object and a \emph{class prototype} indicating the
\emph{presentation type} of the object.  In some cases the application
object and the class prototype can be the same identical object.  In
other cases, they are different.  We refer to this class prototype as
the \emph{presentation prototype}. 

Take for instance an \emph{information system} that displays
application objects such as \emph{people} and \emph{organizations}.
These objects are instances of corresponding application model classes
such as \texttt{person} and \texttt{organization}.  A presentation of
a person would typically contain that person both as the application
object and as the presentation prototype.  In other cases, the system
might display information about the \emph{salary} of people, or the
total number of \emph{hours} spent on a project.  In these cases, the
application object might in both cases be a \cl{} \texttt{number}, but
to distinguish them, the presentation prototype might be an instance
of the class \texttt{salary} in one case and \texttt{duration} in the
other case.

Ultimately, only the application itself can decide how an object
should be displayed as graphics.  A \emph{person}, for instance, might
be displayed as a silhouette of a person with a name underneath.  

When the function \texttt{accept} is called, it is given the name of a
presentation type and an optional predicate that can be used to
further discriminate possible input.  When the pointer enters a
presentation, the presentation prototype is checked against the
presentation type given to \texttt{accept}, and if it is the right
type, then the optional predicate is applied to the underlying
application object.  If the predicate returns \emph{true}, then the
application is a candidate for satisfying the call to
\texttt{accept}.


