\chapter{Future development}

\section{Short-term improvements}

This section contains a 'to-do' list of short-term possible
improvements.  The feasibility of some of these improvements has not
been established, so they might not be possible or practical. 

\begin{itemize}
\item Document that \code{vbrick} is a special case of \code{vframe}
  and \code{hbrick} is a  special case of \code{hframe} and check
  beforehand that this is actually true. 

\item The setters for the height and the width of a zone should
  probably be in the \code{clim-ext} package.  Client code that needs
  to change the dimensions of a zone should set the sprawls instead. 

\item When the sprawls of a zone change for \emph{any} reason, i.e.,
  it could be that they needed to be recalculated because the sprawls
  of the children changed, or it could be that they were explicitly
  modified, then the notify-child-[hv]sprawl-changed should be
  called.  We must therefore make it an :after (I think) method on
  (setf clim3-ext:[hv]sprawl) as opposed to (setf clim3:[hv]sprawl)
  which I think is currently the case.  (setf clim3:[hv]sprawl) should
  be used only by client code to explicitly modify the sprawls of a
  zone.  Zones that ignore the sprawls of their children and impose
  their own should also allow for explicit modification of the
  sprawls.  Zones that compute the sprawls from the children should
  signal an error if an attempt is made to explicitly change the
  sprawls. 

\item Currently we have \code{impose-size}, \code{set-vpos},
  \code{natural-size}.  Add functions \code{(setf clim-ext:width)}
  \code{(setf clim-ext:height)}, \code{natural-width},
  \code{natural-height}.

\item Investigate the possibility of using the MOP to verify that when
  a zone class is finalized, it has a good selection of superclasses,
  so that it reacts to the notification protocols in appropriate ways.

\item Eliminate \code{paint-opaque} and \code{paint-translucent} in
  favor of a single \code{paint-rectangle} (maybe with a better name)
  that works essentially like \code{paint-translucent}.  For
  performance reasons, put a test in the new function to capture the
  important case when the opacity is close to 1. 

\item Currently, \code{paint-opaque} and \code{paint-translucent} take
  a \emph{color} object, whereas \code{paint-pixel} takes the three
  color components.  Choose one method and modify accordingly.
  Perhaps it is better on these low-level functions to go for the
  color components in case a color object is not available. 

\item It looks a bit strange that \code{paint-pixel} does not take any
  position arguments, and instead requires a \code{with-position}
  wrapper.  On the other hand, it is consistent with how
  \code{paint-opaque} and \code{paint-translucent} work.  We could
  either keep it as it is, change \code{paint-opaque} and
  \code{paint-translucent} to also take the area as arguments (bad
  idea, I think) or provide both styles in a different set of
  functions.   For now, probably keep it as it is.

\item Make access to the current theme more convenient, for instance
  by using global symbol macros or by using a \code{with-theme} or
  \code{with-current-theme} macro.

\end{itemize}

\section{Presentation types}

Presentation types are an essential part of \climtwo{} and we think of
\clim{} too.  However there are many aspects of the way presentation
types are implemented in \climtwo{} that are not satisfactory. 

For one thing, in many places it is assumed that the English language
is used for things like prompting, names of types, etc.  The
\emph{options} of presentation types seem to be essentially about
that, so presentation-type options might not be required in \clim{}.  

In \climtwo{}, defining presentation types is supposed to mimic the
way types are defined in \cl{}.  However there is a fundamental
difference between the two.  In \cl{} using \code{deftype} does not
introduce any new types; only new names that expand into existing
types.  In \climtwo{} (and in \clim{} as well), however, defining a
new presentation type means defining a new type.  This fact justifies
the use of classes to represent presentation types, because a new
class is the only way to create a new type in \cl{}.  

An essential aspect of the way presentation types are implemented in
\climtwo{} is that the type is distinct from the object of that type,
allowing any \cl{} object to be labeled with any presentation type.
This functionality might be superfluous, because it seems reasonable
to consider the presented object as a wrapper for a real object.  If
indeed this turns out to \emph{be} reasonable, then normal \cl{}
classes with normal names could be used to represent presentation
types, as opposed to using a list the way \climtwo{} sometimes does.  

In \climtwo{} there is an option for creating presentation types that
indicates that the parameters are other types.  It is used to define
presentation types such as \code{and}, \code{or}, and
\code{sequence}.  This option seems very kludgy.  Perhaps it would be
better to reserve a fixed number of operators with type parameters. 

The \climtwo{} specification states that superclasses have lexical
access to the parameters.  But this ``feature'' complicates things. 

The annotation by Gilbert Baumann in section 23.3.3 of the \climtwo{}
specification indicates a problem.  It has to do with the ``special''
arguments in a \code{define-presentation-generic-function} named
\code{type-key}, \code{type-class}, \code{parameters}, \code{options},
and \code{type}; specifically how these are matched.  

In \climtwo{}, the functions \code{accept} and \code{present} are tied
to the concept of a \climtwo{} \emph{stream}.  I am not sure this
connection is reasonable.  On the other hand the \code{view} type
might be useful. 

The \climtwo{} function \code{accept} is very useful.  The function
\code{present} is more dubious.  I find myself using
\code{with-output-as-presentation} most of the time, just because the
function \code{present} never seems to do what I want. 

In \climtwo{}, a presentation contains a full presentation type,
including parameters.  But this seems silly, just like it is silly to
have to specify the type of the \cl{} object 42.  While we may retain
the fact that the presentation type of an object is independent of the
underlying object itself, it seems reasonable to try to get rid of the
parameters of the type in a presentation.  But doing so \emph{might}
require getting rid of the independence of the underlying object and
the presentation type.
