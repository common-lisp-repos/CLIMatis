\chapter{Zones}

At the lowest level of abstraction, \clim{} uses a \emph{canvas} which
is an axis-aligned rectangular area of opaque \emph{pixels} supplied
by some display server.  Drawing on this canvas means taking some
arbitrary subset of pixels on it, and modifying their color values in
some way.  The result thus depends not only on what was drawn, but
also what was there before.  The \emph{order} in which drawings are
made is thus important.  The canvas uses integer coordinates.

Applications do not draw directly to the canvas, and in fact \clim{}
does not have any representation of the canvas.  Instead,
\clim{} manages a \emph{hierarchy} of \emph{zones}.  A zone is an
axis-aligned rectangular area.  Each zone manages its own abstract
\emph{drawing plane} which is a potentially infinite plane with its
origin in the upper-left corner of the zone.  The position and size
of a zone are defined with integer coordinates, and the position is
relative to its parent zone.  The effect of drawing a zone is
\emph{clipped} by its parent zone, i.e., a request to draw a zone
is accompanied by a clipping region, and zones are not allowed to
draw outside this region, because if they did, the result would be
incorrect. 

When a hierarchy is attached to a \emph{port} (representing a display
server), the \emph{backend} takes this hierarchy of zones and
\emph{realizes} each zone in the hierarchy in some device-specific
way.  Some backends (such as X11) which allow for windows to be
nested, may choose to create a window for each zone.  Other backends
may chose to create a single window for the top zone.

Zones are either \emph{atomic} or \emph{compound}.  An atomic zone
represents an elementary graphics object such as a line, a rectangle,
an ellipse, or an image.  Zones for managing input are also atomic
zones.  A compound zone contains a bunch of children organized in a
way that depends on the specific compound zone.  It could be a \cl{}
list, a \cl{} two-dimensional array, or something else.  Each zone has
a \emph{position}, a \emph{size}, and a \emph{depth}.  The position
and the depth represent information relating to the \emph{parent} of
the zone.  The depths of the children of a compound zone determine the
order in which children are traversed for things like drawing.

In terms of zone composition operations, zones on the same level of
the hierarchy are usually composed using the \emph{compose-over}
operation, in the order determined by the depth, and for a compound
zone, the result of the composition of the children is composed with
the parent, where the parent acts as a clipping region for the
children.  In other words, rendering a composite zone is a matter of
rendering the children in order of decreasing depth, except that the
output is clipped by the parent. 

Most zones are not created with exact sizes, because the exact size of
a zone may depend on the backend.  This is in particular the case for
a \emph{text zone} the size of which depends on the \emph{font} used
to render it, and the font is specific to each backend.  Instead, each
zone has a horizontal and a vertical \emph{sprawl}.  A sprawl is an
object that represents the \emph{desired} preferred, minimum, and
maximum sizes of a zone.  The are just \emph{desired} because each
zone type must be able to handle being assigned any size as decided by
the parent zone or the backend.  Some zones have sprawls assigned at
creation time.  Others, such as many compound layout zones, compute
their sprawls as a function of the sprawls of their children.  Yet
others, in particular \emph{text} compute their sprawls by consulting
the port that it is connected to.

Applications can manipulate the zone hierarchy in several different
ways:  

\begin{itemize}
\item They may explicitly insert or delete zones.  In this case
  \clim{} may try to minimize the areas of the canvas that need to be
  redrawn by re-using pixel colors that are known to be the same as
  last time around the command loop.

\item Simpler application might generate new zone hierarchies every
  time around the command loop.  In this case, \clim{} will have to
  redraw the entire contents of the zone.

\item An application may produce a new collection of children for some
  compound zone each iteration of the command loop such that some or
  all of the new zones are \emph{the same} as the existing ones.
  \clim{} may compare the new collection to the old one, and again
  minimize redrawing.

\item An application may \emph{move} a zone.  Since the canvas uses
  integer coordinates, \clim{} might then be able to move the pixels
  without requiring the application to generate new output.  This is a
  particularly important optimization when \emph{scrolling} is
  required.  Moving a zone is allowed only when the zone does not have
  a parent, or when the parent is a layout zone that allows its
  children to determine their own positions. 
\end{itemize}

\section{Graphic zones}
\label{sec-zones-graphic}

A \emph{graphic zone} is an atomic zone.  When such a zone is
\emph{painted}, some subset of the pixels inside it may have their
color values altered.

Graphic zones are \emph{immutable}, so as to guarantee that the result
of painting one is the same each time it is painted.  If an
application wants to alter the behavior of some graphics zone, say its
color, then the only way is to stick it inside a compound zone (the
simplest one is the \code{wrap} zone \seesec{sec-zones-layout-wrap})
and replace it with a new one with the altered behavior.%
\footnote{The reason for graphic zone to be immutable is so that
  \clim{} can detect a change by comparing two such zones using
  \code{eq}.  When such a comparison yields \emph{true} and if the
  zone has the same size as before, then \clim{} uses this information
  to decide that the result of painting the zone will be the same as
  last time.}

\subsection{Monochrome}
\label{sec-zones-graphic-monocrome}

The \code{monochrome} zone is not intended to be instantiated
directly.  Instead it is meant to be a superclass of other zones that
use a single color.  

\Defclass{monochrome}

\Definitarg{:color}

\subsection{Opaque}
\label{sec-zones-graphic-opaque}

The simplest graphic zone is the \code{opaque} zone.  It ignores the
color values of the existing pixels and paints the entire zone with
some fixed color value that was given when the zone was created.
Since graphic zones are immutable, it is not possible to change this
color value.  The \emph{sprawls} of an opaque zone are the same as
those of a \code{sponge} \seesec{sec-zones-layout-sponge}, i.e., it is
very elastic, both vertically and horizontally, so that it can take on
any size.

\Defclass{opaque}

\Definitarg{:color}

This class is a subclass of \code{monochrome}
\seesec{sec-zones-graphic-monocrome}  As such, it admits the
\texttt{:color} initarg. 

\Defun {opaque} {color}

\subsection{Masked}
\label{sec-zones-graphic-masked}

The \code{masked} zone uses a \emph{color} and a \emph{mask} to alter
the pixels inside it when painted.  A mask is a 2-dimensional array of
opacity values (also known as \emph{alpha} values).  The zone is
painted using the \emph{compose over} color composition operation.
The minimum, preferred, and maximum sizes for a masked zone are the
same as the sizes for the mask.  The mask is considered to be
positioned so that the upper-left corner of the mask coincides with
the upper-left corner of the zone.  If the zone is smaller than the
mask in some dimension, then the output is clipped by the zone as
usual.  If the zone is larger than the mask in some dimension, the
pixels outside the mask are considered completely transparent. 

\Defclass{masked}

\Definitarg{:color}

This class is a subclass of \code{monochrome}
\seesec{sec-zones-graphic-monocrome}  As such, it admits the
\texttt{:color} initarg. 

\Definitarg {:opacities}

\Defun {masked} {color opacities}

\subsection{Translucent}
\label{sec-zones-graphic-translucent}

The \code{translucent} zone uses a \emph{color} and an \emph{opacity}
value (also known as an \emph{alpha} value) to alter the pixels inside
it when painted.  Since graphic zones are immutable, it is not
possible to change the color or the opacity value.  The \emph{sprawls}
of a translucent zone are the same as those of a \code{sponge}
\seesec{sec-zones-layout-sponge}, i.e., it is very elastic so that it
can take on any size.

\Defclass{translucent}

\Definitarg{:color}

This class is a subclass of \code{monochrome}
\seesec{sec-zones-graphic-monocrome}  As such, it admits the
\texttt{:color} initarg. 

\Definitarg {:opacity}

\Defun {translucent} {color opacity}

\subsection{Lighten}
\label{sec-zones-graphic-lighten}

The \code{lighten} zone is a special case of the \code{translucent}
zone \seesec{sec-zones-graphic-translucent}  using the completely
\emph{white} color.  The effect is to make whatever is behind it
lighter.

This zone works by providing a default \texttt{:color} initarg with
the completely white color as a value. 

\Defclass {lighten}

\Definitarg {:opacity}

\Defun {lighten} {opacity}

\subsection{Darken}
\label{sec-zones-graphic-darken}

The \code{darken} zone is a special case of the \code{translucent}
zone \seesec{sec-zones-graphic-translucent}  using the completely
\emph{black} color.  The effect is to make whatever is behind it
darker.

This zone works by providing a default \texttt{:color} initarg with
the completely black color as a value. 

\Defclass {darken}

\Definitarg {:opacity}

\Defun {darken} {opacity}

\subsection{Image}
\label{sec-zones-graphic-image}

The \code{image} zone uses an \emph{image} to alter the pixels inside
it when painted.  An image is a 2-dimensional array of \emph{pixels}.
A pixel is a color and an opacity value.  The zone is painted using
the \emph{compose over} color composition operation.  The minimum,
preferred, and maximum sizes for an image zone are the same as the
sizes for the image.  The image is considered to be positioned so that
the upper-left corner of the image coincides with the upper-left
corner of the zone.  If the zone is smaller than the image in some
dimension, then the output is clipped by the zone as usual.  If the
zone is larger than the image in some dimension, the pixels outside
the image are considered completely transparent.

\Defclass{image}

\Defun {image} {pixels}

\section{Layout zones}
\label{sec-zones-layout}

A \emph{layout zone} is a zone that may have some \emph{children} that
are also zones.  Each type of layout zone is characterized by:

\begin{itemize}
\item the method it uses for combining the sprawls of its children
  into its own vertical and horizontal sprawls, and
\item the method it uses for determining the size and relative
  position of each child, given the size that is imposed on it. 
\end{itemize}

\subsection{Wrap}
\label{sec-zones-layout-wrap}

The \code{wrap} zone is a zone that has either no children or a single
child.  

If the \code{wrap} zone has no child, it behaves in the same way as
the \code{sponge} zone. \seesec{sec-zones-layout-sponge} If it has a
child, then it takes on the sprawls of that child.

If the \code{wrap} zone has a child, then it positions that child at
its own upper-left corner, and it imposes its own vertical and
horizontal size on the child.

The wrap zone is useful as a place holder in the zone hierarchy for
places that are altered in some way by an application.  For example,
if an application wants to change the background color of a gadget,
perhaps as a result of some pointer motion, then the background color
of the gadget would typically be an \code{opaque} zone
\seesec{sec-zones-graphic-opaque} contained inside a \code{wrap}
zone.  To alter the color, the application would then replace the
child of the \code{wrap} zone by a different \code{opaque} zone having
the desired color.

\Defclass {wrap}

\Defun {wrap} {\optional child}

\subsection{Sponge}
\label{sec-zones-layout-sponge}

The \code{sponge} zone is a zone that has either no children or a
single child.  

It ignores the sprawls of its child (if any) and imposes its own.
Each imposed sprawl has zero minimum and preferred size and infinite
maximum size.

If the \code{sponge} zone has a child, then it positions that child at
its own upper-left corner, and it imposes its own vertical and
horizontal size on the child.

\Defclass {sponge}

\Defun {sponge} {\optional child}

\subsection{Vsponge}
\label{sec-zones-layout-vsponge}

The \code{vsponge} zone is a zone that has either no children or a
single child.  If it has no child, it behaves the same way as the
\code{sponge} zone. \seesec{sec-zones-layout-sponge}  If it has a
child, it copies the horizontal sprawl of the child, and it ignores
the vertical sprawl of its child and imposes its own with zero minimum
and preferred size and infinite maximum size.

If the \code{vsponge} zone has a child, then it positions that child
at its own upper-left corner, and it imposes its own vertical and
horizontal size on the child.

\Defclass {vsponge}

\Defun {vsponge} {\optional child}

\subsection{Hsponge}
\label{sec-zones-layout-hsponge}

The \code{hsponge} zone is a zone that has either no children or a single
child.  If it has no child, it behaves the same way as the
\code{sponge} zone. \seesec{sec-zones-layout-sponge}  If it has a
child, it copies the vertical sprawl of the child, and it ignores
the horizontal sprawl of its child and imposes its own with zero minimum
and preferred size and infinite maximum size.

If the \code{hsponge} zone has a child, then it positions that child
at its own upper-left corner, and it imposes its own vertical and
horizontal size on the child.

\Defclass {hsponge}

\Defun {hsponge} {\optional child}

\subsection{Brick}
\label{sec-zones-layout-brick}

The \code{brick} zone is a zone that has either no children or a
single child.  It ignores the sprawls of its child (if it has one) and
imposes its own.  The minimum, preferred, and maximum sizes are all
the same for the vertical sprawl, and all the same for the horizontal
sprawl.  Those sizes are determined when the brick is created.

If the \code{brick} zone has a child, then it positions that child at
its own upper-left corner, and it imposes its own vertical and
horizontal size on the child.

\Defclass {brick}

\Defun {brick} {width height \optional child}

\subsection{Vbrick}
\label{sec-zones-layout-vbrick}

The \code{vbrick} zone is a zone that has either no children or a
single child.  If it has no child, it behaves the same way as the
\code{brick} zone. \seesec{sec-zones-layout-brick}  If it has a
child, it copies the horizontal sprawl of the child, and it ignores
the vertical sprawl of its child and imposes its own with identical
minimum, preferred, and maximum sizes, determined at creation time.

If the \code{vbrick} zone has a child, then it positions that child at
its own upper-left corner, and it imposes its own vertical and
horizontal size on the child.

\Defclass {vbrick}

\Defun {vbrick} {height \optional child}

\subsection{Hbrick}
\label{sec-zones-layout-hbrick}

The \code{hbrick} zone is a zone that has either no children or a
single child.  If it has no child, it behaves the same way as the
\code{brick} zone. \seesec{sec-zones-layout-brick}  If it has a
child, it copies the vertical sprawl of the child, and it ignores
the horizontal sprawl of its child and imposes its own with identical
minimum, preferred, and maximum sizes, determined at creation time.

If the \code{hbrick} zone has a child, then it positions that child at
its own upper-left corner, and it imposes its own vertical and
horizontal size on the child.

\Defclass {hbrick}

\Defun {hbrick} {width \optional child}

\subsection{Vbox}
\label{sec-zones-layout-vbox}

The \code{vbox} zone is a zone that arranges its children in a
vertical box.  It can have an arbitrary number of children, but it is
best suited for a moderate number of children.  When thousands of
children are needed, the \code{vtree} zone is a better
choice. \seesec{sec-zones-layout-vtree}

The vertical sprawl of the \code{vbox} is computed by combining the
vertical sprawls of its children in \emph{series} and the horizontal
sprawl of the \code{vbox} is computed by combining the horizontal
sprawls of its children in \emph{parallel}.
\seesec{sec-zone-protocols-geometry-sprawl}

The vertical size imposed on the \code{vbox} is distributed according
to the method described in \refSec{sec-zone-protocols-geometry-sprawl}
for children combined in series.  The horizontal size imposed on the
\code{vbox} is imposed as is on each child. 

\Defclass {vbox}

\Defun {vbox} {children}

\Defun {vbox*} {\rest children}

\subsection{Hbox}
\label{sec-zones-layout-hbox}

The \code{hbox} zone is a zone that arranges its children in a
horizontal box.  It can have an arbitrary number of children.

The horizontal sprawl of the \code{hbox} is computed by combining the
horizontal sprawls of its children in \emph{series} and the vertical
sprawl of the \code{hbox} is computed by combining the vertical
sprawls of its children in \emph{parallel}.
\seesec{sec-zone-protocols-geometry-sprawl}

The horizontal size imposed on the \code{hbox} is distributed according
to the method described in \refSec{sec-zone-protocols-geometry-sprawl}
for children combined in series.  The vertical size imposed on the
\code{hbox} is imposed as is on each child. 

\Defclass {hbox}

\Defun {hbox} {children}

\Defun {hbox*} {\rest children}

\subsection{Vtree}
\label{sec-zones-layout-vtree}

The \code{vtree} zone is the ideal zone for presenting a scrollable
sequence of \emph{lines} of arbitrary objects.

The \code{vtree} zone is similar in spirit to the \code{vbox} zone
\seesec{sec-zones-layout-vbox} except that the behavior is simpler in
order to provide optimal performance for a very common use case.

Just like the \code{vbox}, the vertical sprawls of the children are
combined in \emph{series} and the horizontal sprawls of the children
are combined in parallel. \seesec{sec-zone-protocols-geometry-sprawl} 

However, the \code{vtree} differs from the \code{vbox} in the way that
it imposes size on its children.  Each child of the \code{vtree}
always take on its natural size, both vertically and horizontally.
Furthermore, the \emph{first} child of the \code{vtree} is always
positioned at the upper-left corner of the \code{vtree} (i.e, at
positio 0,0).  The remaining children are positioned immediately below
the first one, in the natural order.  

Should a vertical size be imposed on the \code{vtree} that is
\emph{greater} than the sum of the natural size of each child, the
extra space will be empty and positioned below the last child.
Similarly, should a horizontal size be imposed on the \code{vtree}
that is \emph{greater} than its natural size, then the extra space
will be empty and positioned to the right of the children.

Should a vertical size be imposed on the \code{vtree} that is
\emph{less} than the sum of the natural size of each child, some
suffix of the children will not be visible.  Similarly, should a
horizontal size be imposed on the \code{vtree} that is \emph{less}
than its natural size, then the output of each child will be clipped
horizontally after the natural size of each child. 

The rules cited above make it impractical to make the \code{vtree}
zone the child of any zone that does not behave essentially like the
\code{scroll} zone or like the \code{vscroll} zone.

The children of the \code{vtree} zone are organized as a balanced
\emph{tree}.  This tree structure uses a number of \emph{internal}
zones that can not be used in a context other than as children of the
\code{vtree} zone.  

From the point of view of application code, the \code{vtree} zone
behaves like an \emph{editable sequence} of children.

\Defun {vtree} {}

Create an empty \code{vtree}, i.e. a \code{vtree} zone with no children.

\Defgeneric {insert} {vtree zone position}

Insert \code{zone} as a child at \code{position} in \code{vtree}.  The
value of \code{position} must be between $0$ and $N$, where $N$ is the
number of children in \code{vtree}.  A value of $0$ means that the
zone becomes the \emph{first} child and a value of $N$ means that the
zone becomes the \emph{last} child in the sequence.

\Defgeneric {find} {vtree position}

Return the child zone at \code{position} in \code{vtree}.  The value
of \code{position} must be between $0$ and $N-1$, where $N$ is the
number of children in \code{vtree}.

\Defgeneric {delete} {vtree position}

Delete the child zone at \code{position} in \code{vtree}.  The value
of \code{position} must be between $0$ and $N-1$, where $N$ is the
number of children in \code{vtree}. 

\subsection{Grid}
\label{sec-zones-layout-grid}

The \code{grid} zone is a zone that arranges its children in a
two-dimensional grid.  

To compute the vertical sprawl of the \code{grid}, the vertical
sprawls of the children of each \emph{row} are first combined in
\emph{parallel}, and the resulting sprawls for each row are then
combined in \emph{series}.  

To compute the horizontal sprawl of the \code{grid}, the horizontal
sprawls of the children of each \emph{column} are first combined in
\emph{parallel}, and the resulting sprawls for each column are then
combined in \emph{series}.
\seesec{sec-zone-protocols-geometry-sprawl}

The vertical size imposed on the \code{grid} is distributed to its
rows according to the method described in
\refSec{sec-zone-protocols-geometry-sprawl} for children combined in
series, and the resulting vertical size for each row is imposed as is
on each child of the row.

The horizontal size imposed on the \code{grid} is distributed to its
columns according to the method described in
\refSec{sec-zone-protocols-geometry-sprawl} for children combined in
series, and the resulting horizontal size for each column is imposed
as is on each child of the row.

\subsection{Pile}
\label{sec-zones-layout-pile}

The \code{pile} zone is a zone that arranges its children on top of
each other.

To compute the vertical sprawl of the \code{pile}, the vertical
sprawls of its children are combined in \emph{parallel}, and to
compute the horizontal sprawl of the \code{pile}, the horizontal
sprawls of its children are also combined in \emph{parallel}. 


The vertical and the horizontal sizes imposed on the \code{pile}
are imposed as is on each child.

\Defclass {pile}

\Defun {pile} {children}

\Defun {pile*} {\rest children}

\subsection{Border}
\label{sec-zones-layout-border}

The \code{border} zone is a zone that positions a child so that it is
surrounded by a an empty space of some thickness which is given at the
time the instance of the \code{border} class is created.

The \code{border} zone always has a single child.  It computes its
sprawls by taking the sprawls of the child, and adding twice the border
thickness to the preferred and minimum size of those sprawls.  If the
maximum size of any of the sprawls is bounded, then twice the border
thickness is also added to the maximum size in order to obtain the
maximum size of the corresponding sprawl of the \code{border} zone.
If the maximum size of any of the sprawls of the child is unbound,
then the maximum size of the corresponding sprawl of the \code{border}
zone is also unbounded.

Notice that the border pixels are completely transparent.  If a
colored border is desired, this zone should be used together with
graphics zone providing the border pixels inside a \code{pile} zone
\seesec{sec-zones-layout-pile}

\Defclass {border}

\Definitarg {:thickness}

The thickness of the border in pixels.  

\Definitarg {:child}

The child zone.

\Defun {border} {thickness child}

Constructor.

\subsection{Scroll}

The \code{scroll} zone is a zone that has either no children or a single
child.  

Either way, it ignores the sprawls of its child and imposes its own,
making it behave exactly like a \code{sponge} zone.  

If the \code{scroll} zone has a child, it does not impose any position
or size of that child.  It always gives the child its \emph{natural
  size}, and it does not modify the position of the child as set by
the application.

\Defclass {scroll}

\Definitarg {:child}

\Defun {scroll} {\optional child}

Constructor. 

\subsection{Vscroll}

The \code{vscroll} zone is a zone that has either no children or a single
child.  

Vertically, the \code{vscroll} zone behaves like the \code{vsponge}
zone in that it ignores the vertical sprawl of its child and instead
imposes its own, which makes it willing to take on any size.  

Horizontally, the \code{vscroll} zone copies the horizontal sprawl of
its child.  If it does not have a child, it behaves like the
\code{sponge} zone.  

If the \code{vscroll} zone has a child, then the horizontal size and
position are imposed on that child so that the left and right edges of
the child align with the left and right edges of the \code{vscroll}
zone.  

If the \code{scroll} zone has a child, it does not impose any vertical
position or vertical size of that child.  It always gives the child
its \emph{natural height}, and it does not modify the vertical
position of the child as set by the application.

\Defclass {vscroll}

\Definitarg {:child}

\Defun {vscroll} {\optional child}

Constructor. 

\subsection{Hscroll}

\Defclass {hscroll}

The \code{hscroll} zone is a zone that has either no children or a single
child.  

Horizontally, the \code{hscroll} zone behaves like the \code{hsponge}
zone in that it ignores the horizontal sprawl of its child and instead
imposes its own, which makes it willing to take on any size.  

Vertically, the \code{hscroll} zone copies the vertical sprawl of
its child.  If it does not have a child, it behaves like the
\code{sponge} zone.  

If the \code{hscroll} zone has a child, then the vertical size and
position are imposed on that child so that the upper and lower edges of
the child align with the upper and lower edges of the \code{hscroll}
zone.  

If the \code{scroll} zone has a child, it does not impose any horizontal
position or horizontal size of that child.  It always gives the child
its \emph{natural width}, and it does not modify the horizontal
position of the child as set by the application.

\Definitarg {:child}

\Defun {hscroll} {\optional child}

Constructor. 

\subsection{Bboard}

\Defclass {bboard}

\Definitarg {:children}

\subsection{Center}

The \code{center} zone has either no children or a single child.  If
it has no children, it behaves like the \code{sponge} zone.

If the \code{center} zone has a child, it copies the sprawls of its
child, except that it sets the maximum size to infinity, making it
infinitely stretchable.  

If the vertical size of the \code{center} zone is greater than or
equal to the \emph{natural height} of its child, then it centers the
child vertically (with a possible error of half a pixel) by
surrounding the child with totally transparent rows of pixels, and it
gives the child its natural height.  If the vertical size of
the \code{center} zone is strictly smaller than the natural height of
its child, it imposes its own height on the child.

Similarly, if the horizontal size of the \code{center} zone is greater than or
equal to the \emph{natural width} of its child, then it centers the
child horizontally (with a possible error of half a pixel) by
surrounding the child with totally transparent columns of pixels, and it
gives the child its natural width.  If the horizontal size of
the \code{center} zone is strictly smaller than the natural width of
its child, it imposes its own width on the child.

\Defclass {center}

\Definitarg {:child}

\Defun {center} {\optional child}

Constructor. 

\subsection{Top}

\Defclass {top}

\Definitarg {:child}

\subsection{Bottom}

\Defclass {bottom}

\Definitarg {:child}

\subsection{Left}

\Defclass {left}

\Definitarg {:child}

\subsection{Right}

\Defclass {right}

\Definitarg {:child}

\subsection{Top-left}

\Defclass {top-left}

\Definitarg {:child}

\subsection{Top-right}

\Defclass {top-right}

\Definitarg {:child}

\subsection{Bottom-left}

\Defclass {bottom-left}

\Definitarg {:child}

\subsection{Bottom-right}

\Defclass {bottom-right}

\Definitarg {:child}

\section{Input zones}
\label{sec-zones-input}

\subsection{Motion}
\label{sec-zones-input-motion}

The \code{motion} zone is used to track the position of the pointer,
expressed in coordinates relative to the motion zone.  A
\emph{handler} is given when the motion zone is created, and that
handler is called whenever the position of the pointer changes,
whether it is outside or inside the zone.  The handler is called with
three arguments: the zone, the relative horizontal position, and the
relative vertical position. 


%%  LocalWords:  superclass initarg scrollable
