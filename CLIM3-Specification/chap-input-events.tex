\chapter{Input}
\label{chap-input-events}

Input events are handled in two different ways; either globally for
the application or by some zone.  

Currently, two types of input processing are done by zones, namely
detecting when the pointer enters or leaves some zone
\seesec{sec-zones-input-visit} and determining the position of the
pointer. \seesec{sec-zones-input-motion}

Keyboard events and pointer button events are handled globally.

\section{Handling keyboard events}

Two generic functions are involved in handling keyboard events:

\begin{itemize}
\item \code{handle-key-press}
\item \code{handle-key-release}
\end{itemize}

These two generic functions are called whenever the corresponding
input event is detected.  These functions take tree arguments:

\begin{itemize}
\item A \emph{handler} object.
\item The \emph{key code} of the key involved.
\item The \emph{modifier state} at the time the event happened.
\end{itemize}

The handler object should be an instance of the class
\code{key-handler}.  The user may create subclasses of this classes,
and add methods on these generic functions specialized for those
subclasses.

The \emph{key code} and the \emph{modifier state} are backend specific
values.  Users can write backend-specific handlers, but \clim{}
provides functionality for turning these values into
backend-independent objects.

Whenever the backend detects a key-press event, it calls the function
\code{handle-key-press} passing it the value of the special variable
\code{*key-handler*}, the key code, and the modifier state. 

Similarly, whenever the backend detects a key-release event, it calls
the function \code{handle-key-release}, again passing it the value of
the special variable \code{*key-handler*}, the key code, and the
modifier state.

The global value of the variable \code{*key-handler*} is the global
value of another variable named \code{*null-key-handler*} which
contains an instance of the class \code{null-key-handler}.  A default
method on \code{handle-key-press} specialized for the
\code{null-key-handler} class exists, and it does nothing.  Similarly,
a default method on \code{handle-key-release} specialized for the
\code{null-key-handler} class exists, and it does nothing.

As mentioned above, \clim{} provides functionality for turning the
\emph{key code} and the \emph{modifier state} into backend-independent
objects.  For this purpose, \clim{} provides a function
\code{standard-key-processor}.  This function calls the generic
function \code{port-standard-key-processor} which has methods for each
backend.  The function \code{standard-key-processor} takes a \emph{key
  code} and a \emph{modifier} and returns a \cl{} \emph{character}
corresponding to the two backend-specific arguments. 

\Defclass {read-keystroke-handler}

\Definitarg {:receiver}

This initarg should be a function of one
argument, a \emph{character}.  A method exists on
\code{handle-key-press} specialized for \code{read-keystroke-handler}
which calls the receiver function with a character that corresponds to
the key that was pressed.  The corresponding method on
\code{handle-key-press} does nothing.  To transform the key-press
event into a character, the method on \code{handle-key-press} calls
the function \code{standard-key-processor} discussed above.

\section{Handling pointer button events}

Two generic functions are involved in handling pointer button events:

\begin{itemize}
\item \code{handle-button-press}
\item \code{handle-button-release}
\end{itemize}

These two generic functions are called whenever the corresponding
input event is detected.  These functions take tree arguments:

\begin{itemize}
\item A \emph{handler} object.
\item The \emph{button code} of the button involved.
\item The \emph{modifier state} at the time the event happened.
\end{itemize}

The handler object should be an instance of the class
\code{button-handler}.  The user may create subclasses of this class,
and add methods on these generic functions specialized for those
subclasses.

The \emph{button code} and the \emph{modifier state} are backend
specific values.  Users can write backend-specific handlers, but
\clim{} provides functionality for turning these values into
backend-independent objects. 

Whenever the backend detects a button-press event, it calls the function
\code{handle-button-press} passing it the value of the special variable
\code{*button-handler*}, the button code, and the modifier state. 

Similarly, whenever the backend detects a button-release event, it calls
the function \code{handle-button-release}, again passing it the value of
the special variable \code{*button-handler*}, the button code, and the
modifier state.

The global value of the variable \code{*button-handler*} is the global
value of another variable named \code{*null-button-handler*} which
contains an instance of the class \code{null-button-handler}.  A
default method on \code{handle-button-press} specialized for the
\code{null-button-handler} class exists, and it does nothing.  Similarly,
a default method on \code{handle-button-release} specialized for the
\code{null-button-handler} class exists, and it does nothing.

\Defclass {visit}

The \code{visit} class is a class that can be used to mix into zone
classes in order to create a zone that detects when the pointer enters
or leaves the zone.  

When the pointer enters a zone with this class as a superclass, it
calls the generic function \code{enter} (see below) with the zone as
an argument, and when the pointer leaves a zone with this class as a
superclass, it calls the generic function \code{leave} (see below)
with the zone as an argument.

\Definitarg {:inside-p}

This initarg takes a function of two arguments, the zone-local $x$ and
$y$ coordinates of the pointer, and returns a Boolean value indicating
whether the pointer is to be considered \emph{inside} the zone.  The
default value for this initarg is a function that always returns
\emph{true}. 

\Defgeneric {enter} {zone}

This generic function is called when the pointer enters a zone that
has the class \code{visit} as a superclass.  It supplies the zone as
an argument.  This generic function uses the \code{progn} method
combination.

\Defgeneric {leave} {zone}

This generic function is called when the pointer leaves a zone that
has the class \code{visit} as a superclass.  It supplies the zone as
an argument.  This generic function uses the \code{progn} method
combination.

\Defclass {clickable}

This class can be used to mix into zone classes in order to create a
zone that is sensitive to pointer clicks.  Notice that this class is
itself not a subclass of \code{zone}. 

\Defgeneric {attention} {zone}

This generic function is called in order to make a zone sensitive to
pointer clicks.  The argument should be an instance of
\code{clickable}.  The class \code{clickable} supplies a method on
this generic function that replaces the button handler by a button
handler that calls the generic function \code{button-press} (see
below) when a pointer button is pressed inside the zone, and that
calls the generic function \code{button-release} (see below) when a
pointer button is released inside the zone

If the application wants a zone to become sensitive to pointer clicks
when the pointer enters the zone, then it should make the zone a
subclass of both \code{visit} and \code{clickable}, and it should
supply a method on \code{enter} that calls \code{attention}. 

\Defgeneric {at-ease} {zone}

This generic function is called in order to make a zone insensitive to
pointer clicks.  The argument should be an instance of
\code{clickable}.  The class \code{clickable} supplies a method on
this generic function that restores the original button handler.

If the application wants a zone to become insensitive to pointer clicks
when the pointer leaves the zone, then it should make the zone a
subclass of both \code{visit} and \code{clickable}, and it should
supply a method on \code{leave} that calls \code{at-ease}. 

\Defgeneric {button-press} {zone button}

This generic function is called when a pointer button is pressed
inside a zone that is a subclass of \code{clickable}, provided that
the generic function \code{attention} has previously been called on
that zone.  The first argument is the zone and the second argument is
the result of calling \code{standard-button-processor} on the button
code and the modifiers, i.e. a list where the first element is a
keyword symbol corresponding to the button and the remaining elements
are keyword symbols representing modifiers.

\Defgeneric {button-release} {zone button}

This generic function is called when a pointer button is released
inside a zone that is a subclass of \code{clickable}, provided that
the generic function \code{attention} has previously been called on
that zone.  The first argument is the zone and the second argument is
the result of calling \code{standard-button-processor} on the button
code and the modifiers, i.e. a list where the first element is a
keyword symbol corresponding to the button and the remaining elements
are keyword symbols representing modifiers.


%%  LocalWords:  backend superclass initarg
